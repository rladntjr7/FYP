\chapter{Conclusion}
This project aimed to develop a method for sperm tracking using machine learning and computer vision approaches. The main objectives were to design and build a sperm detection model with YOLOv8 and to utilize the Kalman filter to track the individual sperm to assess motility. The results showed that the proposed method achieved high accuracy and usability. Future work can focus on improving the model performance and versatility by creating a more diverse dataset, adding morphology detection features, enhancing model performance, and developing other applications of this program. 

Many claims that 2023 will be the inflection point for AI development. The increased public interest in popular AI tools, like ChatGPT, has brought more rapid development of other AIs in many fields. Although some are worrying about the rapid development of AI for its unknown effects on society, the emergence of more powerful AI models is inevitable and imminent. As a mechanical engineering student, I did not have any extensive knowledge about AI prior to this project. I am grateful that I was able to study this topic as my final year project before opening a new chapter in my life.